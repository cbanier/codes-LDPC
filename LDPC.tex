\documentclass{article}

\usepackage[T1]{fontenc}
\usepackage[utf8]{inputenc}
\usepackage[french]{babel}
\usepackage{hyperref}
\usepackage{graphicx}

\hypersetup{
    colorlinks=true,
    linkcolor=blue,
    filecolor=magenta,
    urlcolor=cyan,
}
\urlstyle{same}

\title{Les Codes LDPC }
\author{Corentin Banier - Maher Karboul}
\date{}

\begin{document}

\maketitle
\tableofcontents

\section{Introduction}
Les codes LDPC (Low Density Parity Check) sont des codes linéaires correcteurs d'erreurs qui assurent la transmission d'informations .

\subsection{Codes correcteurs}
\subsection{Matrice de parité}
\subsection{Matrice génératrice}
\subsection{Syndrome}

\section{Comment ça se fabrique un code LDPC?}
\section{L'algorithme de décodage}
\section{Expériences}

\end{document}