\documentclass{article}

\usepackage[T1]{fontenc}
\usepackage[utf8]{inputenc}
\usepackage[french]{babel}
\usepackage{hyperref}
\usepackage{graphicx}

\hypersetup{
    colorlinks=true,
    linkcolor=blue,
    filecolor=magenta,
    urlcolor=cyan,
}
\urlstyle{same}

\title{Les Codes LDPC }
\author{Corentin Banier - Maher Karboul}
\date{}

\begin{document}

\maketitle
\tableofcontents

\section{Introduction}
La conception des codes LDPC binaires avec un faible poids d'erreurs demeurre un problème non entiérement résolu. Les codes LDPC (Low Density Parity Check) sont des codes linéaires correcteurs d'erreurs qui assurent la transmission d'informations. Ils forment une classe de codes en bloc qui se caractérisent par une matrice de contrôle creuse. Ils ont été décrits pour la première fois dans la thèse de Gallager au début des années 60. Dans ce travail , on va reprendre l'algorithme de Gallager et on va l'implémenter et l'analyse afin de l'optimiser au maximum.\newline
Le rapport est organisé de la manière suivante . La section 2 présente les étapes de fabrication d'un code LDPC ainsi qu'une matrice de contrôle qui répond à des conditions bien spécifiques. La section 3 présente l'algorithme détaillé de décodage LDPC. La section 4 est dédiée aux expériences pratiques qu'on a effectué durant l'implémentation de l'algorithme. La section 5 est une conclusion sur le projet.

\subsection{Codes correcteurs}
Lors de la transmission d'une information , des erreurs peuvent se produire. Cette problématique de corrections des erreurs de transmission est très importante dans notre monde connecté, qu'il s'agisse des communications entre ordinateurs par internet, des conversations téléphoniques etc.. \newline
Un code correcteur, souvent désigné par le sigle anglais ECC (Error-correcting code), est une technique de codage basée sur la redondance. \newline
Un code est une application ijective $\Phi:\{0,1\}^k \rightarrow \{0,1\}^n $. \newline
Le paramètre k est appelé la \textbf{dimension} du code $\Phi$ et le paramètre n est appelé la \textbf{longueur} du code : on dit que $\Phi$ est un code de paramètres (k,n). \newline
Soit $\Phi$ un code d'image C . \newline
On appelle \textbf{capacité de correction} de $\Phi$ le plus grand entier $e_c$ tel qu'on soit toujours capable de corriger $e_c$ erreurs ou moins. \newline
On appelle \textbf{distance minimale} de $\Phi$ et on note $d_c$ la plus petite distance non nulle entre deux mots de code. \newline
On a \textbf{$e_c = $} $\frac{d_c-1}{2}$
Parmi les exemples de codes correcteurs qu'on peut citer: code de répétition, code carré et le code de Hamming
\subsubsection{Code de répétition}
Le code de répétition se résume par transmettre le message deux fois pour s'assurer contre les erreurs. Par exemple, Alice veut transmettre un mot de quatre bits à Bob. 
\begin{tabbing}
\hspace{5cm}\textbf{m=0111} \newline
\end{tabbing}
Elle va donc envoyer le mot codé
\begin{tabbing}
    \hspace{5cm}\textbf{c=01110111}
\end{tabbing}
le mot reçu par Bob sera noté y. Par exemple, si Bob reçoit
\begin{tabbing}
    \hspace{5cm}\textbf{y=01110110}
\end{tabbing}
Il peut constater qu'une erreur, au moins, s'est produite. Il peut dire que l'erreur est soit sur le quatrième ou le huitième bit.
\subsubsection{Code carré}
\subsubsection{Code de Hamming}
\subsection{Matrice de parité}
\subsection{Matrice génératrice}
\subsection{Syndrome}

\section{Comment ça se fabrique un code LDPC?}
\section{L'algorithme de décodage}
\section{Expériences}
\section{Conclusion}

\end{document}