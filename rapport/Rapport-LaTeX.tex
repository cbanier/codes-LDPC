\documentclass[french,nochapter,11pt]{rapportUB}  
% Options french (1-) ou english (2-)
%    1- si vous rédigez en français (chargement du package [french]{babel})
%    2- si vous rédigez en anglais (chargement du package [english]{babel})
% Options chapter (1-) ou nochapter (2-)
%    1- la commande \chapter peut être utilisée dans le document (la classe report est chargée)
%    2- la commande \chapter NE peut PAS être utilisée dans le document (la classe article est chargée)
% Options 10pt ou 11pt ou 12pt (taille des caractères)
% Option nologo (à utiliser seulement si vous ne souhaitez pas afficher le logo de l'université sur la première page / le logo doit être dans le même dossier que votre fichier tex et s'appeler logo (l'extension n'a pas d'importance)
%
% Les options par défaut sont french, nochapter,11pt

\college{Collège Sciences et Technologies}
\uf{UF Mathématiques et Interactions - Informatique}
\program{Licence Mathématiques Informarique} %A remplir
\course{Projet tutoré} %A remplir
\academicyear{2020 -- 2021} %A remplir
\author{ %A remplir
  Corentin Banier
  \\
  Maher Karboul
}

%Paquetages additionnels
\usepackage{placeins}
\usepackage{multirow}
\usepackage{framed}
%Pour gérer la bibliographie
\usepackage[backend=bibtex,style=authoryear]{biblatex}
\addbibresource{biblio.bib}
\DeclareDelimFormat{nameyeardelim}{\addcomma\space}

%Déclaration d'un nouvel opérateur
\DeclareMathOperator*{\argmin}{arg\,min}
\DeclareMathOperator*{\argmax}{arg\,max}


\begin{document}

\title{Codes LDPC}

\maketitle

\begin{center}
\tableofcontents %affichage de la table des matières
\clearpage
\end{center}

%-----------------------------------------------------
% IMPORTANT : A décommenter pour ajouter l'engagement de non plagiat au rapport.
%\nonplagiat{Prenom1 Nom1}[Prenom2 Nom2][Prenom3 Nom3]
%-----------------------------------------------------


\section{Introduction}
\label{sec:introduction}


La figure
\ref{fig:logo} présente le logo de l'université.

\begin{figure}[!h]
\centering
\includegraphics[scale=0.1]{logo.eps}  
\caption{Le logo de l'université}
\label{fig:logo}
\end{figure}


L'état de l'art de
\textcite{Applegate2006} sur le problème de voyageur de commerce est une référence sur ce sujet.

\vspace{0.8cm}
Dans la suite du document, la section \ref{sec:formalisation} présente la formalisation du problème. Dans la section \ref{sec:modelisation}, nous modélisons celui-ci comme un programme linéaire en nombre entiers. Nous introduisons plusieurs algorithmes de résolutions du problème dans la section \ref{sec:algo} ....


\section{Formalisation du problème}
\label{sec:formalisation}

Le problème se définit de la mnière suivante ....

\section{Modélisation du problème}
\label{sec:modelisation}
Un programme linéaire en nombre entiers pour le problème de sac-à-dos est le suivant :

\begin{align}
\max~ &  \sum_{i=1}^{n}p_ix_i \label{eq_obj} \\
sc~ & \sum_{i=1}^{n}w_ix_i \leq W \label{eq_ct1} \\
& x_i \in \{0,1\} & \forall i=1...n \label{eq_ct2} 
\end{align}

L'objectif \eqref{eq_obj} consiste à maximiser le profit associé aux objets sélectionnés. 
La contrainte \eqref{eq_ct1} prend en compte la capacité du sac lors de la sélection des objets.
La contrainte \eqref{eq_ct2} définit le domaine des variables de décisions.


\section{Algorithmes de résolution}
\label{sec:algo}

Pour écrire des algorithmes, le paquetage suivant est très utile : \url{http://tug.ctan.org/macros/latex/contrib/algorithm2e/doc/algorithm2e.pdf}  


\section{Expérimentations et résultats}
\label{sec:exp}

La Table \ref{table:resultats} présente un résumé des résultats de nos expérimentations. 

\begin{table}[htbp]
  \centering
  \caption{Comparaison des différentes méthodes}
  \label{table:resultats}
    \begin{tabular}{l|rrrrr}
    \hline
    \multirow{2}[0]{*}{Algorithme}&  \multicolumn{2}{c}{Temps CPU (ms)}  & \multicolumn{2}{c}{Gap MS} &  \multirow{2}[0]{*}{\#Données MS}  \\
        & \multicolumn{1}{c}{Moyenne}   & \multicolumn{1}{c}{Ecart type} & \multicolumn{1}{c}{Moyenne} & \multicolumn{1}{c}{Ecart type} & \\
          \hline
    Heuristique PPV & 29 & 96 & 8.4\% & \textbf{36\%} &  6/38\\
    Heuristique MI & 16984 & 78093  & \textbf{1.3\%} & 42\% &   \textbf{30/38} \\
    \hline
    \end{tabular}%
\end{table}%

\FloatBarrier

\section{Conclusion}
\label{sec:conclusion}
C'est la fin.


%% Les appendices commencent après la commande \appendix
%% Les commandes \chapter (si l'option chapter est sélectionnée), \section, \subsection peuvent s'utiliser comme habituellement
%\appendix

%\clearpage % à utiliser pour afficher la biblographie sur une nouvelle page
% References
\printbibliography


\end{document}
